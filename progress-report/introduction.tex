\section{Introduction}
In recent years processors have been running into limits in single-threaded processing speed \cite{cpu_clk_speed}. Overall performance has been increased by taking advantage of parallelism. This can be achieved using multiple cores to gain general parallelism or through hardware accelerators specifically designed to exploit the parallelism inherent in an algorithm. An example would be multiplying matrices. Many matrix operations involve computing many simpler calculations on the components of the matrix. These calculations are often independent and can be performed in parallel. A processor would compute these sequentially but a specific hardware implementation would be able to compute them in parallel.

An FPGA allows hardware to be developed and iterated on much faster than other options such as using an ASIC or logic ICs. They allow hardware to be designed with HDL code that is then constructed on the FPGA \cite{whatisanfpga}. For this reason, they will be used in this project to implement the accelerator. Some FPGA boards have a built-in processor but the board used in this project does not so that will also be constructed on the FPGA.

RISC-V is a good option for this processor as it is an open standard instruction set architecture \cite{risc} that can be implemented by anyone and has multiple options for FPGA implementations such as Rocket Chip \cite{rocketchip}. It is also used extensively in other products such as high-performance SiFive boards and the low-power Espressif's ESP32-C3 microcontroller. Because of this adoption, more libraries are available such as a port of Debian.

\newpage
\subsection{Objectives}

Develop a hardware accelerator for a RISC-V processor communicating over the AXI bus and compare it against a bare-metal software solution.

\begin{itemize}
	\item Generate a RISC-V core on an FPGA and run bare-metal code on it (Must).
	\item Design an IP block that connects to the RISC-V's AXI bus that can send and receive data (Must).
	\item Design an IP block that performs the hardware-accelerated functionality (Must).
	\item Develop a driver in bare-metal C code to use the accelerator (Must).
	\item Design and implement a test for the accelerator (Must).
	\item Implement the test using a software approach and compare results (Should).
	\item Implement a hardware accelerator using an alternative method of communication such as an APB bus and compare (Could)
	\item Run Linux on the processor and make the accelerator available for use (Won't)
\end{itemize}

\newpage

\subsection{Related Work}
\subsubsection{Work A - Implementation of a RISC-V Processor with Hardware Accelerator \cite{risc-v_hard_accel}}
This paper covers the development of a hardware matrix multiplier for a RISC-V processer. They used a ZedBoard which uses the same Artix-7 FPGA chip as the Nexys A7 100T that this project uses. They use PULPino \cite{pulpino}, a 32-bit single-core RISC-V microprocessor. This project does not support the Nexys A7 100T FPGA used in our project so vivado-risk-v \cite{vivado-risk-v} is used to generate a 64-bit RISC-V core using the Rocket Chip project \cite{rocketchip}. This takes more of the resources available to the FPGA but can be reduced to 32-bit if needed as we will not be running Linux. They used the APB bus to communicate between the accelerator and PULPino microprocessor due to its interface being simpler than the AXI bus. In our project, the AXI interface will be used as it is recommended by the base project and provides the option for higher-performance transfers. When tested their accelerator reduced the number of clock cycles required from 603 to 134. This is a good improvement but due to the processor having a clock speed of 5 MHz, it is unable to compete with modern processors. They encountered many errors when trying to build C programs for the PULPino and required much trial and error. The process for compiling code for our project has proved much simpler.

\subsubsection{Work B - Matrix Multiplication Accelerator \citep{matrix_mult_accel}}
This paper also used an accelerator for matrix multiplication but with a DE1-SoC board using a built-in ARM Cortex-A9 HPS as the processer and a Cyclone V FPGA. They used parallel ports to communicate between the HPS and FPGA \cite{pio}. Two approaches were used based on the naive method. The first was to store the input matrices in two registers and use massive parallelization to set the values in the output register. This allowed values to be accessed with zero cycles of latency and in parallel but is very hard to scale. The other method used M10k memory blocks to store the inputs. These only have a single read port and write port limiting the ability to exploit parallelism so they opted to improve performance by pipelining and increasing clock speeds. Their tests showed that the HPS performed better than their accelerator but performed significantly more consistently than the HPS. The accelerator and HPS ran at very different clock speeds, 100MHz and 925MHz. As the tests measured runtime in ms this would have heavily benefited the HPS. A clock cycle test as used in "Implementation of a RISC-V Processor with Hardware Accelerator" may have provided more comparable results.

\subsubsection{Comparison}
\begin{table}[H]
	\centering
	\resizebox{\textwidth}{!}{%
		\begin{tabular}{c|c|c|}
			\cline{2-3}
			& \begin{tabular}[c]{@{}c@{}}Implementation of a RISC-V Processor\\ with Hardware Accelerator\end{tabular}                                      & Matrix Multiplication Accelerator                                                                                                                                                                                                    \\ \hline
			\multicolumn{1}{|c|}{\begin{tabular}[c]{@{}c@{}}Development Board\\ Used\end{tabular}} & \begin{tabular}[c]{@{}c@{}}ZedBoard - Artix-7 and \\ ARM Cortex-A9 MPcore\end{tabular}                                                        & \begin{tabular}[c]{@{}c@{}}DE1-SoC - Cyclone V and \\ ARM Cortex-A9\end{tabular}                                                                                                                                                     \\ \hline
			\multicolumn{1}{|c|}{Processor Used}                                                   & PULPino (RISC-V) on FPGA                                                                                                                      & ARM Cortex-A9 HPS                                                                                                                                                                                                                    \\ \hline
			\multicolumn{1}{|c|}{Communication Method}                                             & APB bus                                                                                                                                       & Parallel Port                                                                                                                                                                                                                        \\ \hline
			\multicolumn{1}{|c|}{Algorithm Used}                                                   & Naive method with parallelism                                                                                                                 & \begin{tabular}[c]{@{}c@{}}Naive method using massive parallelism and \\ "single-threaded" with heavy  pipelining\end{tabular}                                                                                                       \\ \hline
			\multicolumn{1}{|c|}{Testing}                                                          & \begin{tabular}[c]{@{}c@{}}Measures clock cycles required to complete\\  a 4x4 matrix multiplication in \\ hardware and software\end{tabular} & \begin{tabular}[c]{@{}c@{}}Measured runtime (ms) to compute 20x20, 40x40, \\ 60x60 and 80x80 matrix multiplications \\ in hardware and software\\ Measured power usage (W) for hardware and \\ software implementations\end{tabular} \\ \hline
		\end{tabular}%
	}
	\caption{Comparison Table of Related Works}
	\label{tab:comp}
\end{table}

Both works use development boards with a HPS but work A is using the PULPino microcontroller implemented on the FPGA instead. This is a lower-performance processor but allows a greater level of control over the design. They use the APB bus to attach their accelerator while work B uses a parallel port that is built into the chip to connect the HPS and FPGA.

Work A implements a 4x4 matrix accelerator exploiting parallelism. This method is not scalable in hardware due to the exponential increase in hardware required as the matrix size increases. Block matrix multiplication can be used to split larger matrices into a size that is compatible with the accelerator decreasing the impact of this issue. Work B also looked at this method but decided to use a more scalable design by creating a single module that calculates one value of the output matrix and makes heavy use of pipelining to make this process more efficient. This takes more steps to complete a calculation but can handle far bigger matrices. For this project, the method used will need to be evaluated for how important scalability is compared to performance.

Work B has tested their design with 4 different matrix sizes 20, 40, 60 and 80 while work A has used only the 4x4 matrix that is directly compatible with their accelerator. They have used different methods of testing, work A uses clock cycles to measure execution time. This allows for excellent comparisons between the hardware and software approaches as it removes the impact of any difference in clock rates. Work B uses runtime in ms which showed lower performance for their accelerator than software implementation. The HPS used runs with a clock speed 9.25x faster than the accelerator giving it an advantage in this test which may explain the results. Work B also tested power usage which showed the accelerator was more power efficient. For this project clock cycles seem to be a better method of performance testing but power usage also provides interesting data about how beneficial an accelerator could be.